\documentclass[11pt,a4paper,twoside]{article}
\usepackage[utf8]{inputenc}
\usepackage[english]{babel}
\usepackage{blindtext}


\begin{document}

\noindent {\bfseries\LARGE The Journey of Karl Koala\par}
\bigskip\noindent
\blindtext

\section{Level 1: Definitions \& Concepts}

1.  What is a simple example of a non-positional number system used historically?
2.  What does the term "positional number system" mean?
3.  What is the base of the Decimal number system?
4.  What is the base of the Binary number system, and why is it fundamental for modern computers?
5.  What digits are used in the Binary number system (Base-2)?

\section{Level 2: Basic Identification \& Place Value}

6.  In the decimal number $123_{10}$, what value does the digit '2' represent?
7.  What is the decimal value of the binary number $101_2$?
8.  What digits are used in the Hexadecimal system (Base-16)? List them all.
9.  In the hexadecimal number $1A5_{16}$, what is the decimal value of the digit 'A'?
10. What is the base of the Octal number system, and what digits does it use?

\section{Level 3: Simple Conversions (Binary/Decimal)}

11. Convert the decimal number $13_{10}$ to binary.
12. Convert the binary number $1101_2$ to decimal.
13. Convert the decimal number $25_{10}$ to binary.
14. Convert the binary number $10110_2$ to decimal.
15. What is the largest decimal number that can be represented with 4 binary digits (bits)?

\section{Level 4: Hexadecimal \& Octal Concepts \& Conversions}

16. Convert the hexadecimal number $2B_{16}$ to decimal.
17. Convert the decimal number $45_{10}$ to hexadecimal.
18. Convert the octal number $37_8$ to decimal.
19. Convert the decimal number $20_{10}$ to octal.
20. In the hexadecimal number $FACE_{16}$, what place value does the 'A' represent in terms of powers of 16?

\section{Level 5: Inter-base Conversions (Binary/Hex/Octal)}

21. Convert the binary number $11010110_2$ directly to hexadecimal.
22. Convert the hexadecimal number $D5_{16}$ directly to binary.
23. Convert the binary number $10111001_2$ directly to octal.
24. Convert the octal number $153_8$ directly to binary.
25. Convert the hexadecimal number $A8_{16}$ to octal (Hint: Use binary as an intermediate step).

Compare the number of digits necessary to represent the following decimal numbers in
binary, octal, decimal, and hexadecimal representations. You need not determine the
actual representations -- just the number of required digits. For example, representing
the decimal number 12 requires four digits in binary (1100 is the actual representation),
two digits in octal (14), two digits in decimal (12), and one digit in hexadecimal (C).
a) 8
b) 60
c) 300
d) 1000
e) 999,999

\end{document}
